
\documentclass[9pt]{beamer}

\usepackage[utf8]{inputenc}
\usepackage{colortbl}
\usepackage[english]{babel}

\newcommand{\myblue} [1] {{\color{blue}#1}}
\newcommand{\newauthor}[4]{
  \parbox{0.26\textwidth}{
    \texorpdfstring
      {
        \centering
        #1 \\
        \myblue{{\href{#2}{\texttt{#3}}}} \\
        #4 \\
      }
      {#1}
  }
}



% beamer template
\beamertemplatetransparentcovereddynamic
\usetheme{default}

\hypersetup{%
  pdftitle={Why Golang looks like that},%
   pdfauthor={Tomasz Grodzki},%
%
}

\title[Why Golang looks like that]{Why Golang looks like that}
\author[Tomasz Grodzki]{
 \parbox{0.26\textwidth}{
	\texorpdfstring
	  {
		\centering
 		Tomasz Grodzki \\
 		Co-Founder \& Developer, AlphaSOC \\
 		\myblue{\href{mailto:tomasz@alphasoc.com}{\texttt{tomasz@alphasoc.com}}} \\
 		\myblue{\href{https://github.com/tg}{\texttt{https://github.com/tg}}} \\
 	  }
	{Tomasz Grodzki}
}
 }

\date{2018-12-10}

\begin{document}

\frame{\titlepage
}

\part<presentation>{Main Talk}

\section[slides]{slides}

\begin{frame}[fragile]
\frametitle{Long time ago...}


Java and C++ were used for writing servers (at least at Google).


They requires too much bookkeeping and repetition.


Some programmers moved towards more dynamic languages (Python).


Efficiency and type safety at stake!



\end{frame}

\begin{frame}[fragile]
\frametitle{Birth}


Born out of frustration.


Efficient compilation, efficient execution, or ease of programming?


First sketches in September 2007 (Robert Griesemer, Rob Pike, Ken Thompson).


Plenty of siblings (Rust, Elixir, Swift).



\end{frame}

\begin{frame}[fragile]
\frametitle{Language goals}


\begin{itemize}
\item Easy: compiled, statically typed, but with a dynamic feel.
\item Modern: support for networked and multicore computing.
\item Fast: at most a few seconds to build a large executable on a single computer.
\end{itemize}

"Go was intended as a language for writing server programs that would be easy to maintain over time." (golang.org)


"Clean procedural language designed for scalable cloud software." (Rob Pike)



\end{frame}

\begin{frame}[fragile]
\frametitle{Code design principles}


\begin{itemize}
\item Reduce clutter and complexity.
\item Everything is declared exactly once.
\item Syntax is clean and light on keywords.
\item No hierarchy: types just are, they don't have to announce their relationships.
\item Orthogonality of concepts.
\item Orthogonality of features.
\end{itemize}


\end{frame}

\begin{frame}[fragile]
\frametitle{Where are my features?}


Go does not compete on features.


As of Go 1, the language is fixed.


Features add complexity and hurt readability.


Simple features that interact in simple ways.


Simple means readable and predictable.


Simple means easy to maintain.



\end{frame}

\begin{frame}[fragile]
\frametitle{Simplicity is complicated}


Great talk by Rob Pike: "Simplicity is complicated". (I've borrowed a few slides.)


Go is actually complex, but it \emph{feels} simple.


Interacting elements must mesh seamlessly, without surprises.


Requires a lot of design, implementation work, refinement.


Simplicity is the art of hiding complexity.



\end{frame}

\begin{frame}[fragile]
\frametitle{A few simple things in Go}


\begin{itemize}
\item garbage collection
\item goroutines
\item constants
\item interfaces
\item packages
\end{itemize}

Each hides complexity behind a simple facade.



\end{frame}

\begin{frame}[fragile]
\frametitle{(Simple) Garbage collecting}


Super simple to use:
- no interface
- not event mentioned in the spec


Very difficult to implement well.


Shouldn't affect language goals, e.g. compilation time.



\end{frame}

\begin{frame}[fragile]
\frametitle{(Simple) Concurrency}


Start a routine with a \texttt{go} keyword:



\begin{verbatim}
go function(args...)

\end{verbatim}


Fire and forget.


Zero management:


\begin{itemize}
\item no stack size
\item no routine ID
\item no return
\end{itemize}


\end{frame}

\begin{frame}[fragile]
\frametitle{(Simple) Constants}


Constants are just numbers.


They don't have specific type.


Integers and floating points have "infinite" precision.


Can be used more freely than variables:



\begin{verbatim}
x := 2.0         // x is float64
fmt.Println(2*x) // prints "4"

y := 2.0 * time.Second // y is time.Duration (int64)
fmt.Println(30 * y)    // prints "1m0s"

z := 2.5 * time.Second // error: constant 2.5 truncated to integer

\end{verbatim}



\end{frame}

\begin{frame}[fragile]
\frametitle{(Simple) Interfaces}


Interface is just a set of methods. No data.



\begin{verbatim}
type Reader interface {
    Read([]byte) (int, error)
}

\end{verbatim}


Statically and dynamically checked:



\begin{verbatim}
var r Reader = os.Stdin       // statically checked
var x interface{} = os.Stdin  // statically checked
r = x.(Reader)                // dynamically checked

\end{verbatim}


More general than OOP.


Implement Read/Write and take advantage of \myblue{\href{https://golang.org/pkg/io/}{\texttt{package io}}}.



\end{frame}

\begin{frame}[fragile]
\frametitle{(Simple) Packages}



\begin{verbatim}
package http
...
import "net/http"

\end{verbatim}


Two keywords – \texttt{package}, \texttt{import}.


Took a long time to design.


Affects program design, syntax, naming, building, linking, testing, ...


Separation of package path from package name enabled \texttt{go} \texttt{get}.



\end{frame}

\begin{frame}[fragile]
\frametitle{Syntax}


\end{frame}

\begin{frame}[fragile]
\frametitle{Why is the syntax so weird?}


Only declaration syntax is different (from C-like languages).


Designed to be easy to analyze.


Can be parsed without symbol table.


Much easier to build tools.



\end{frame}

\begin{frame}[fragile]
\frametitle{C syntax}


In C there is not special syntax for declarations.


You write an expression involving the item being declared:



\begin{verbatim}
int x;
int f(int a, int b);
int (*fp)(int a, int b);

\end{verbatim}


If you write ``(*fp)(a, b)`` you'll call a function that returns int, hence \texttt{fp} is a pointer to a function.


Gets complicated quite quickly:



\begin{verbatim}
int (*fp)(int (*f)(int x, int y), int b)

\end{verbatim}



\end{frame}

\begin{frame}[fragile]
\frametitle{Go syntax}


Distinct type syntax in declaration.


Similar to many languages outside of the C family.



\begin{verbatim}
x int
p *int
a [3]int

\end{verbatim}


Reads nicely from left to right.


No direct correspondence between the look and use.



\end{frame}

\begin{frame}[fragile]
\frametitle{Works well for functions functions functions}


Simple example is similar to C:



\begin{verbatim}
func main(argc int, argv []string) int

\end{verbatim}


Reads well: function main takes an int and a slice of strings and returns an int.


More complex declarations are also readable:



\begin{verbatim}
f func(func(int,int) int, int) int
f func(func(int,int) int, int) func(int, int) int

\end{verbatim}



\end{frame}

\begin{frame}[fragile]
\frametitle{Who is weird now?}



\begin{verbatim}
f func(func(int,int) int, int) func(int, int) int

\end{verbatim}


vs



\begin{verbatim}
int (*(*fp)(int (*)(int, int), int))(int, int)

\end{verbatim}


Go's declarations read left to right.


C's read in a spiral? See \myblue{\href{http://c-faq.com/decl/spiral.anderson.html}{\texttt{Clockwise/Spiral Rule.}}}



\end{frame}

\begin{frame}[fragile]
\frametitle{Invisible semicolon}


Semicolons are in the formal grammar.


Good for parsers, not for people.


Injected automatically, without lookahead.


Lack of lookahead has some side effects:
	- opening brace of a function cannot appear on a line by itself,
	- fixed style,
	- easier for tools and interactive implementation of Go.



\end{frame}

\begin{frame}[fragile]
\frametitle{References}


\begin{itemize}
\item \myblue{\href{https://www.youtube.com/watch?v=rFejpH\_tAHM}{\texttt{Rob Pike - Simplicity is complicated}}}
\item \myblue{\href{https://golang.org/doc/faq}{\texttt{golang.org/doc/faq}}}
\end{itemize}


\end{frame}

\end{document}
